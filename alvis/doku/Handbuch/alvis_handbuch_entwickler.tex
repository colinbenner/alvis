\documentclass[10pt,a4paper,oneside]{scrbook}
\usepackage[utf8x]{inputenc}
\usepackage{ucs}
\usepackage{amsmath}
\usepackage{amsfonts}
\usepackage{amssymb}
\usepackage{makeidx}

\usepackage{listings}
\usepackage{graphicx}
\usepackage{inputenc}
\usepackage[german]{babel}
\usepackage[T1]{fontenc}
\usepackage{hyperref}
\renewcommand{\familydefault}{\sfdefault}
\usepackage{helvet}
\setcounter{secnumdepth}{4}

\author{Jan Bauerdick,
Colin Benner,
Eduard Boos,
Dominik Dingel,\\
Simon Groth,
Sebastian Schmitz, 
Frank Weiler}
\title{Alvis\\Handbuch für Entwickler}
\makeindex
\begin{document}
\maketitle
\newpage
\tableofcontents
\newpage
Wir wünschen viel Spaß bei der Weiterentwicklung von Alvis.
\chapter{Benötigtes Equipment}
\section{Software}
\subsection{Eclipse}
Alvis als RCP Projekt kann am besten mit Eclipse weiterentwickelt werden, wir haben mit mindestens der Version 3.6 gearbeitet.

Es besteht die Möglichkeit bei eclipse.org das Paket ,,Eclipse for RCP and RAP Developers'' herunterzuladen, dies ist ein komplett vorkonfiguriertes und mit allen Abhängigkeiten bestücktes Eclipse.

Falls ein bereits vorhandenes Eclipse für die Entwicklung von Alvis benutzt werden soll, kann man die benötigten Plug-ins auch über die Update Funktion von Eclipse beziehen.
\subsection{SVN}
Für die Versionskontrolle und zur zentralen Ablage benutzen wir ein zur Verfügung gestelltes SVN. Entsprechendes Plugins und Programme sind für Eclipse und jedes Betriebssystem verfügbar. 

\section{Kenntnisse}
Für das (Weiter-)Entwickeln der Plug-ins für Alvis sind Kenntnisse im Umgang mit RCP erforderlich. Diese sollte man sich aneignen, bevor man mit der Entwicklung von einem Plug-in beginnt. Dabei waren folgende Quellen für uns hilfreich.

\begin{itemize}
\item Vogella.de - Sehr zu empfehlen für Einsteiger. \\
Lars Vogels Tutorial für Eclipse RCP bietet einen guten Einstieg in die Welt von RCP. Anfänger sollten hier beginnen.\\
http://www.vogella.de/articles/EclipseRCP/article.html
\item RalfEbert.de - RCP Buch für fortgeschrittene. \\
Ralf Ebert hat ein Buch über RCP in Form einer Webseite veröffentlicht. In diesem Buch werden in jedem Kapitel andere Konzepte von RCP erklärt. \\
http://www.ralfebert.de/eclipse\_rcp/
\item Einstieg in Eclipse 3.5 - Künneth, Thomas \\
Wer lieber etwas in der Hand hat, der ist mit diesem Buch gut bedient. Hier wird Eclipse RCP gut erläutert. \\
Verfügbar in der Uni-Bibliothek, ISBN: 978-3-8362-1428-5
\end{itemize}


\newpage
\chapter{Alvis}
\section{Übersicht über die Komponenten}
Bei den einzelnen Komponenten wird die Bezeichnung ,,Plug-in'' vielfältig eingesetzt. Wir sprechen von einem Plug-in für Alvis, wenn wir eine neue Algorithmen Art hinzufügen, oder das Hauptplug-in um andere Funktionalitäten erweitern. Dabei ist Alvis selbst auch nur ein Plug-in.

Alle Plug-ins befinden sich im SVN - Ordner ,,trunk/alvis''.
\subsection{Das Alvis Plug-in}
,,de.unisiegen.informatik.bs.alvis'' - Das Haupt Plug-in, welches die graphische Oberfläche beinhaltet.
\subsubsection{Extension Points}
\begin{itemize}
 \item ,,de.unisiegen.informatik.bs.alvis.extensions.runpreferences''\\
 \item ,,de.unisiegen.informatik.bs.alvis.runvisualizer''\\
 \item ,,de.unisiegen.informatik.bs.alvis.export''\\
 \item ,,de.unisiegen.informatik.bs.alvis.extensionpoints.datatypelist''\\
 \item ,,de.unisiegen.informatik.bs.alvis.extensionpoints.fileextension''\\
	  An diesem Extension Point registrieren sich Plug-Ins, die eine Algorithmen-Art
	  bereitstellen, dies wird durch ihre Dateiendung gekennzeichnet. Dazu muss das Interface
	  de.\-unisiegen.\-informatik.\-bs.\-alvis.\-extension\-points.\-IFileExtension im jeweiligen
	  Plug-In implementiert werden.
\end{itemize}

\subsection{Plug-ins für Alvis}
Diese Plug-ins erweitern das Haupt Plug-in um notwendige Funktionen. Die Teilung des Hauptprogramms in verschiedene Plug-ins fördert die Modularisierung und die Wiederverwendbarkeit, ggf. auch für neue Projekte.
\\
Dabei können die Plugins in optionale Plugins und nicht optionale Plugins unterschieden werden. Optionale Plugins sind nicht zwingend zum Ausführen von Alvis erforderlich, während nicht optionale Plugins Kernbestandteile von Alvis sind.
\\ \\
Die Plug-ins befinden sich im Ordner ,,trunk/alvis/plugins''.
\\ \\
\textbf{nicht optionale Plug-ins:}
\begin{itemize}
\item ,,de.unisiegen.informatik.bs.alvis.primitive'' - Primitive.
\item ,,de.unisiegen.informatik.bs.alvis.primitive.datatypes'' - Primitive Datentypen.
\item ,,de.unisiegen.informatik.bs.alvis.ui.navigator'' - Navigator (Package Explorer).
\item ,,de.unisiegen.informatik.bs.alvis.vm'' - Virtuelle Maschine.
\end{itemize}
\textbf{Optionale Plug-ins:}
\begin{itemize}
\item ,,de.unisiegen.informatik.bs.alvis.vm.test'' - Tests der virtuellen Maschine.
\item ,,de.unisiegen.informatik.bs.alvis.editors.xml'' - XML Editor.
\item ,,de.unisiegen.informatik.bs.alvis.help''
\item ,,de.unisiegen.informatik.bs.alvis.graph.help''
\item ,,de.unisiegen.informatik.bs.alvis.sync.help''
\end{itemize}
\subsection{Graph Plug-in}
Das Graph Plug-in erweitert Alvis um die Datentypen Graph, Knoten und Kante.
\begin{itemize}
\item ,,de.unisiegen.informatik.bs.alvis.graph'' - Graph Plug-in.
\item ,,de.unisiegen.informatik.bs.alvis.graph.datatypes'' - Graph Datentypen.
\end{itemize}
\subsection{Betriebssysteme Plug-in}







\subsection{Virtual Machine Plug-In}
Die Virtual Machine steuert den Ablauf eines Algorithmus. Sie ist für die Verwaltung der einzelnen
Schritte der Threads verantwortlich.

\subsubsection{Klassen}
Im Folgenden werden die einzelnen Klassen kurz erläutert, alle Einzelheiten sind in der jeweiligen Java-Doc zu finden.
\begin{itemize}
 \item AbstractAlgo\\
	Definiert die zu implementierende Schnittstelle der Algorithmen für Verwaltungsfunktionen wie: Starten, Stoppen, Inspizieren, Breakpoints
 \item Activator\\
 \item Algo\\
	Ist dabei ein Beispiel Objekt das AbstractAlgo implementiert, aber ohne eine besondere Funktionalität.
 \item AlgoThread\\
	Verwaltet die Algorithmen und dazugehörigen Threads als eine Einheit, die Virtuelle Maschine arbeitet nur mit AlgoThreads.
 \item AlvisFileObject\\
    Nimmt den zu kompilierenden Algorithmus in seiner CharSequence auf und wird an den Compiler, der von ToolProvider
    bereitgestellt wird, überreicht.
 \item BPListener\\
	Die Breakpoint Schnittstelle, beim Erreichen eines Breakpoints wird der entsprechende Listener als eigener Thread aufgerufen.
 \item DPListener\\
	Die Decision Point Schnittstelle, beim Erreichen eines DecisionPoints wird der entsprechende Listener aufgerufen um die Situation zu klären.
 \item DynaCode\\
    Ist für das Kompilieren und Laden des Java-Algorithmus verantwortlich.
    Dazu erzeugt sie aus den benötigten Paketen den Classpath und erzeugt eine Liste von Verzeichnissen, in denen
    Algorithmen liegen. Danach wird der Compiler in Javac mit den erzeugten Parametern angestoßen. Nach dem Kompilieren
    holt sich DynaCode das class-Objekt und lädt es.
 \item Javac\\
    Kompiliert Java-Dateien, die ihm von DynaCode gegeben werden. Dazu baut die Klasse nur ein Array mit den
    Optionen für den Compiler auf und übergibt sie ihm. Zuerst wird versucht, von ToolProvider einen Compiler zu
    beziehen. Im Erfolgsfall wird der Algorithmus in einem AlvisFileObject gespeichert und kompiliert.
    Falls kein Compiler zurückgegeben wird, wird der Compiler aus\\com.\-sun.\-tools.\-javac.\-Main benutzt.
    Dieser Fallback ist nicht schön und soll noch aus Alvis verschwinden. Leider trat das Problem erst gegen Ende der
    Projektgruppe auf. Da es einen funktionierenden Workaround gibt, wurde der Fehler als weniger wichtig erachtet und
    andere Dinge vorgezogen.\\
    Wenn sich tools.jar nicht im Classpath befindet, stoppt Alvis beim Aufruf der Funktion, die etwas
    aus com.\-sun.\-tools.\-javac.\-Main benutzt.
 \item VirtualMachine \\
	Das Schnittstellen Objekt zur GUI, diese arbeitet nur auf der VirtualMachine.
\end{itemize}
\subsection{Die Help-Plug-Ins}
Die Help-Plug-Ins sind dazu da, dem User Hilfestellung zu leisten. Man erreicht sie in Alvis über Hilfe->Hilfeinhalte.\\
Der Aufbau gestaltet sich wie folgt:\\
Plug-Ins müssen den Extension Point org.eclipse.help.toc implementieren und dort eine Help Table of Contens bereitstellen.
Der Pfad zu dieser TOC muss \$nl\$/NameDerTOC.xml lauten, damit die Hilfe in mehreren Sprachen zur Verfügung gestellt werden kann.
Die deutschen Hilfeseiten - und somit auch die deutsche Table of Contents - befinden sich alle im Ordner nl/de, die 
standardmäßig angezeigte, englische direkt im Plug-In. Konvention ist, für HTML-Seiten einen Unterordner html anzulegen,
um die Übersicht zu bewahren.

\newpage
\chapter{Export}
\section{Erste Schritte}
Wir werden den Export des Programms und der Plug-ins über eine Update Seite und über Features bewältigen. Dazu empfehlen wir das Tutorial von Ralf Ebert zur Update Funktionalität in RCP.
\\ \\
http://www.ralfebert.de/blog/eclipsercp/p2\_updates\_tutorial\_36/
\\ \\
Neben den Projekten aus dem vorherigen Kapitel benötigen wir nun auch die Feature-Projekte aus dem Ordner ,,trunk/export''.

\section{Export des Hauptprogramms für verschiedene Plattformen}
Nach Änderungen, Erweiterungen des Hauptprogramms oder der Plugins wollen wir diese jetzt an die Benutzer verteilen. Falls Alvis noch nicht installiert wurde reicht es Alvis neu zu exportieren und auf der Webseite zum Download anzubieten.

Benutzer die Alvis schon installiert haben sollen über die Updatesite eine Möglichkeit  bekommen Alvis zu aktualisieren, im folgenden gehen wir auf die einzelnen Schritte dafür ein.

\subsection{Vorbereitung}
Aus dem Tutorial wissen wir, dass unser Export auf Basis von Features läuft. Diese Features müssen wir nun erstmal aus dem SVN in den Workspace auschecken, in dem auch unser Hauptprogramm und die benötigten Plug-ins liegen.
\\ \\
Die Features befinden sich im Ordner ,,trunk/alvis/export''.
Wir benötigen:
\begin{itemize}
\item ,,org.eclipselabs.p2.rcpupdate''
\item ,,org.eclipselabs.p2.rcpupdate.utils''
\item ,,de.unisiegen.informatik.bs.alvis.app''
\end{itemize}

Das Hauptprojekt wird in einen Ordner exportiert. Dabei wird auch ein Ordner ,,repository'' erstellt. Wir erstellen zuerst einen Ordner, welchen wir für den Export benutzen möchten. Hier ,,D:/Alvis/''. Anschließend noch einen Ordner ,,repository'' für die Update Seite. Hier ,,D:/Alvis/repository''.
\\ \\
Damit die Änderungen auch als Updates auf die Webseite kommen, brauchen wir die vorhandene Update Seite. Diese befindet sich im Ordner ,,trunk/alvis/updates'' und wird in regelmäßigen Abständen automatisch auf den Server kopiert. 
\\ \\
Dazu checken wir den Ordner ,,updates'' aus. Wir wählen ,,D:/Alvis/repository'' als Speicherort.
\subsection{Export mit einer Target Platform}
Um eine RCP Anwendung für verschiedene Plattformen zu exportieren muss die sogenannte ,,Target Platform'' angepasst werden. Mit einer Target Platform kann man festlegen, welche Plug-ins der Anwendung zur Verfügung stehen. Hat man keine spezielle Target Platform gewählt, so bedient sich die Anwendungen aus den Plug-ins, die der installierten Eclipse Distribution zur Verfügung stehen.
\\
\\
Durch den Wechsel der Target Platfrom lassen sich Eclipse Plugins die für andere Betriebssysteme benötigt werden auswählen. So kann mit Eclipse Alvis Cross Plattform gebaut werden.

\subsection{Einstellen der Target Platform}
Wir wollen nun die vorhandene Target Platform als Target Platform für unser Projekt einstellen. Dazu müssen wir das Projekt\\
,,de.unisiegen.informatik.bs.alvis.app'' in unserem Workspace haben. Wir öffnen die Datei ,,allPlatforms.target'' und warten bis alle Plug-ins geladen wurden. Dies kann unter Umständen einige Zeit in Anspruch nehmen. 
\\ \\
Das vorläufige Ergebnis sieht folgendermaßen aus.
\begin{center}
% \includegraphics[scale=0.48]{images/export_all_platforms_00}
\end{center}

Mit einem Klick auf ,,Set as Target Platform'' lässt sich diese Konfiguration einstellen. 
Danach wird der Workspace neu gebaut. Die Projekte sollten keine Fehler aufweisen. Sollten Fehler bezüglich nicht aufzufindender Imports sein ist dies ein Indiz für fehlende Plug-ins in der Target Platform. Über den Button ,,Edit'' lassen sich benötigte Plug-ins aus der Update Seite für Helios holen.
\\ \\
Zum Zeitpunkt des Verfassens wurde Alvis unter Eclipse 3.6 entwickelt. Sollte Alvis inzwischen mit einer neueren Version von Eclipse entwickelt werden, so ist es möglicherweise notwendig auch die Target Platform aufzufrischen. Eine genauere Anleitung findet man im RCP-Buch von Ralf Ebert im Kapitel 16 - Export und Verteilung von RCP-Anwendungen\footnote{http://www.ralfebert.de/eclipse\_rcp/distribution\/}. Noch genauer geht Ebert in seinem Blog\footnote{http://www.ralfebert.de/blog/eclipsercp/rcp\_builds\/} auf die Problematik ein. Es muss, in dem Fenster wo wir uns die Plug-ins aus der Update Seite holen, unbedingt ein Haken bei ,,Include all environments'' gesetzt sein.
\\ \\

Nachdem die Target Platform übernommen wurde und wir die Korrektheit des Programms überprüft haben können wir das Projekt exportieren. 
\subsection{Export}
Wir möchten nochmals darauf hinweisen, dass es nötig ist die Versionsnummern zu erhöhen, damit ein Update auf die jetzt exportierte Software funktioniert. Es müssen sowohl in der \\
,,MANIFEST.MF'' und der \\
,,*.product'', im Hauptprogramm, und in der \\
,,feature.xml'' im Feature ,,~.alvis.app'', \\
die Versionsnummer auf einen gemeinsamen neuen Wert erhöht werden.
\\ \\
Um das Hauptprogramm zu exportieren öffnen wir die Datei ,,build.product'' aus dem Plug-in ,,de.unisiegen.informatik.bs.alvis''.

\begin{center}
% \includegraphics[scale=0.65]{images/export_main_product_00}
\end{center}

Wir klicken auf den  Link ,,\underline{Eclipse Product export wizard}'' und es erscheint der Export Wizard.

\begin{center}
% \includegraphics[scale=0.48]{images/export_main_product_01}
\end{center}

Wir wählen als ,,Roo\underline{t} directory:'' einen beliebigen, noch nicht im Ordner ,,D:/Alvis/'' vorhandenen Ordnernamen und stellen ,,D:/Alvis/'' als ,,Directory:'' ein. 
Wichtig ist, dass der Haken bei ,,Generate metadata repository'' gesetzt ist.
Außerdem setzen wir einen Haken bei ,,Export for multiple platforms''.\\ \\
Wir bestätigen die obigen Einstellungen und klicken auf ,,Next''.
\begin{center}
% \includegraphics[scale=0.48]{images/export_main_product_02}
\end{center}
Im nächsten Fenster wählen wir die Plattformen aus, die wir bedienen möchten.
Nachdem wir auf ,,Finish'' geklickt haben und der Export abgeschlossen ist, können wir im Zielverzeichnis die Ordner für die einzelnen Plattformen finden.
\\ \\
Beim Export wurden im Ordner ,,repository'' die Java Archive generiert, die eine ältere Version von Alvis bei einem Klick auf ,,Help'' \(\rightarrow \) ,,Check for Updates'' für ein Update auf unsere neue Version benötigt. Als Standard Update Pfad ist für Alvis die ,,Alvis Update Seite\footnote{http://alvis.bs.informatik.uni-siegen.de/update/}'' eingetragen also muss der Inhalt des Ordners ,,repository'' in diesen Ordner kopiert werden.

Dies geschieht über einen ,,commit'' der Update Seite.

\end{document}